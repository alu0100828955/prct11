\documentclass{beamer}
\usepackage[utf8]{inputenc}
\usepackage{graphicx}

\newtheorem{definicion}{Definición}
\newtheorem{ejemplo}{Ejemplo}

%%%%%%%%%%%%%%%%%%%%%%%%%%%%%%%%%%%%%%%%%%%%%%%%%%%%%%%%%%%%%%%%%%%%%%%%%%%%%%%
\title[Presentación sobre el Número $\pi$]{Presentación sobre el Número $\pi$ }
\author[Carlos Herrera Carballo]{Carlos Herrera Carballo}
\date[23-04-2014]{23 de abril de 2014}
%%%%%%%%%%%%%%%%%%%%%%%%%%%%%%%%%%%%%%%%%%%%%%%%%%%%%%%%%%%%%%%%%%%%%%%%%%%%%%%

\usetheme{Madrid}
%\usetheme{Antibes}
%\usetheme{tree}
%\usetheme{classic}

\definecolor{pantone254}{RGB}{122,59,122}
\definecolor{pantone3015}{RGB}{0,88,147}
\definecolor{pantone432}{RGB}{56,61,66}
\setbeamercolor*{palette primary}{use=structure, fg=white,bg=pantone254}
\setbeamercolor*{palette secondary}{use=structure, fg=white,bg=pantone3015}
\setbeamercolor*{palette tertiary}{use=structure, fg=white,bg=pantone432}
%%%%%%%%%%%%%%%%%%%%%%%%%%%%%%%%%%%%%%%%%%%%%%%%%%%%%%%%%%%%%%%%%%%%%%%%%%%%%%%
\begin{document}
  
%++++++++++++++++++++++++++++++++++++++++++++++++++++++++++++++++++++++++++++++
\begin{frame}

  
  \titlepage

  \begin{small}
    \begin{center}
     Facultad de Matemáticas \\
     Universidad de La Laguna
    \end{center}
  \end{small}

\end{frame}
%++++++++++++++++++++++++++++++++++++++++++++++++++++++++++++++++++++++++++++++

%++++++++++++++++++++++++++++++++++++++++++++++++++++++++++++++++++++++++++++++
\begin{frame}
  \frametitle{Índice}
  \tableofcontents[pausesections]
\end{frame}
%++++++++++++++++++++++++++++++++++++++++++++++++++++++++++++++++++++++++++++++


\section{Primera Sección}


%++++++++++++++++++++++++++++++++++++++++++++++++++++++++++++++++++++++++++++++
\begin{frame}

\frametitle{¿Qué es el número $\pi$?}

\begin{definicion}
Es la relación entre la longitud de una circunferencia y su diámetro, en geometría euclidiana. Es un número irracional y una de las constantes matemáticas más importantes. Se emplea frecuentemente en matemáticas, física e ingeniería.
\end{definicion}

\end{frame}
%++++++++++++++++++++++++++++++++++++++++++++++++++++++++++++++++++++++++++++++

\section{Segunda Sección}

%++++++++++++++++++++++++++++++++++++++++++++++++++++++++++++++++++++++++++++++
\begin{frame}

\frametitle{Relación entre las funciones trigonométricas y el número $\pi$}

\begin{definicion}
Mediante el uso de las funciones trigonométricas seno y tangente se puede desarrollar una  demostración elemental de la existencia del numero $\pi$ , así como el cáluclo aproximado del mismo.
\end{definicion}

\end{frame}
%++++++++++++++++++++++++++++++++++++++++++++++++++++++++++++++++++++++++++++++

\section{Fórmulas}

\begin{frame}

\frametitle{Fórmulas}

\textbf{La primera fórmula:}

\[x=\frac{a_2 x^2 + a_1 x + a_0}{1+2z^3},\quad x+y^{2n+2}=\sqrt{b^2-4ac}
\]

\pause

\textbf{La segunda fórmula:}
\[\int_{x=0}^{\infty} x\,\text{e}^{-x^2}\text{d}x=\frac{1}{2},\quad\text{e}^{i\pi}+1=0 \]

\pause

\textbf{La tercera fórmula:}

\[\min_{1\le x\le 2}\left(x+\frac{1}{x}\right)=2,\quad \lim_{x\to\infty}\left(1+\frac{1}{x}\right)^x = \text{e}^x \]

\pause

\textbf{La cuarta fórmula:}

\[\Vert x \Vert_2=1, \vert -7 \vert = 7,m|n, m\mid n, <x,y>, \langle x, y\rangle\]

\pause
\end{frame}
\begin{frame}
\textbf{La quinta fórmula:}

\[ \sqrt 2 = 1+\frac{1}{2+\frac{1}{2+\frac{1}{2+\frac{1}{\ddots}}}} \]

\pause

\end{frame}


%++++++++++++++++++++++++++++++++++++++++++++++++++++++++++++++++++++++++++++++
\section{Bibliografía}
%++++++++++++++++++++++++++++++++++++++++++++++++++++++++++++++++++++++++++++++
\begin{frame}
  \frametitle{Bibliografía}

  \begin{thebibliography}{10}

    \beamertemplatebookbibitems
    \bibitem[Aula virtual ULL]{guia}
    
    {\small $http://campusvirtual.ull.es/1314/pluginfile.php/197721/mod_resource/content/3/TeoriaLaTeX.2.pdf$}

    \beamertemplatebookbibitems
    \bibitem[URL: WIKIPEDIA]{latex}
    WIKIPEDIA. {\small $http://www.wikipedia.com/$}

  \end{thebibliography}
\end{frame}

%++++++++++++++++++++++++++++++++++++++++++++++++++++++++++++++++++++++++++++++
\end{document}